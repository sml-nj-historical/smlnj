\documentclass{article}
\usepackage{alltt}
   \newcommand{\nowhere}{{\sf nowhere}}
   \newcommand{\Nowhere}{{\sf Nowhere}}
   \newcommand{\KW}[2]{\newcommand{#1}[1]{{\bf #2}\ }}
   \newcommand{\END}{{\bf end}} 
   \newcommand{\OF}{{\bf of}} 
   \KW{\FUN}{fun} \KW{\VAL}{val}  
   \KW{\LOCAL}{local} \KW{\ANDALSO}{andalso}
   \KW{\CASE}{case} \KW{\LET}{let} \KW{\WHERE}{where}
   \KW{\INCLUDE}{include} \KW{\SIGNATURE}{signature} \KW{\DATATYPE}{datatype}
   \newcommand{\SIG}{{\bf sig}}
   \KW{\IN}{in}
   \KW{\AND}{and}

   \title{\Nowhere{} 1.0 Manual}
   \author{Allen Leung\thanks{New York University. Email:{\tt leunga@cs.nyu.edu}}}
\begin{document}
   \maketitle
\section{Introduction}
If you are like me, you lament the fact that SML doesn't have conditional
pattern matching.  Personally, I don't fully understand the rationale for its
omission in the standard--- even though there are semantics ambiguity 
issues if guard expressions have side effects, conditional pattern matching
is just too useful in practice.   When writing non-trivial transformations, I
find myself having to manually factor pattern matching code, 
whose job should be the compiler's.
I consider its lack in SML a big flaw in an otherwise great language.

  As a remedy, \nowhere{} is a simple source to source tool that translates
SML programs augmented with where-clauses into normal SML code, i.e.,
code without where-clauses\footnote{Thus the name of the tool.}.  

\subsection{Syntax}
Where-clauses that we recognize are of two forms.
In case statements:
\begin{alltt}
    \(pat\) \WHERE \(exp\) => \(exp\)
\end{alltt}
and in function clauses:
\begin{alltt}
    \(pat\sb{1}\) \ldots \(pat\sb{n}\) \WHERE (\(exp\)) = \(exp\)
\end{alltt}
Note that parentheses are not optional in the latter form, because
of syntactic ambiguity with \verb|=|.

For example, the expression
\begin{alltt}
   \CASE [1,2,3] \OF 
      [x,y,z] \WHERE x>y \ANDALSO y>z => 1
    | [x,y,z] \WHERE x<y \ANDALSO y<z => 2
    | _ => 3
\end{alltt}
evaluates to 2.

\Nowhere takes input files of the following form:
\begin{alltt}
{\bf local}
    {\sl datatype definitions}
{\bf in}
    {\sl code}
{\bf end}
\end{alltt}
It looks for where-clauses within the {\sl code} section, and rewrite it into
real SML code.  The {\sl datatype definitions} section is for the benefit
of \nowhere, since it is rather stupid at the moment and doesn't understand
SML's scoping rule and environments.  
All datatypes that are used within the {\sl code}
section should be mentioned here, so that \nowhere{} can find its definitions.

You can use the special file import declaration:
\begin{alltt}
  {\bf include} "filename"
\end{alltt}
to import the definitions from a file.  Note that \nowhere{} will look 
inside nested structure or signature definitions.  So if all datatypes are
properly defined in signature files, all you have to do is import these
files.

\subsection{Semantics}
   Expressions evaluated inside where-clauses should be side-effect free.
This is a built-in assumption in the match compiler.  The result may
not be as expected otherwise.

\section{An Example}
Here's a larger example taken from the x86 peephole phase in MLRISC.
We only show the push and pop folding rules.  The first rule
folds the instructions 
\begin{verbatim}
   subl 4, %esp
   movl src, 0(%esp)
\end{verbatim}
into \verb|push src|, assuming src is not \verb|%esp|. 
The second rule folds
\begin{verbatim}
   movl 0(%esp), dst
   addl 4, %esp
\end{verbatim}
into \verb|pop dst|, again assuming dst is not \verb|%esp|.
\begin{alltt}
\LOCAL\ 
   include "x86Instr.sml" (* import instruction definition *)
\IN\ 
   {\sl ...}
   \FUN loop(current, instrs) =
       \CASE current \OF
         {\sl ...}
      | I.BINARY\{binOp=I.SUBL, src=I.Immed 4, dst=I.Direct dst_i\}:: 
        I.MOVE\{mvOp=I.MOVL,src,dst=I.Displace\{base,disp=I.Immed 0,...\}\}
        ::rest 
           \WHERE C.sameColor(base, C.esp) andalso
                  C.sameColor(dst_i, C.esp) andalso
                  not(isStackPtr src) =>
           loop(rest, I.PUSHL src::instrs)
               
      | I.MOVE\{mvOp=I.MOVL, 
              src=I.Displace\{base, disp=I.Immed 0, ...\}, dst\}::
        I.BINARY\{binOp=I.ADDL, src=I.Immed 4, dst=I.Direct dst_i\}:: 
        rest 
           \WHERE C.sameColor(base, C.esp) andalso
                 C.sameColor(dst_i,C.esp) andalso
                 not(isStackPtr dst) =>
           loop(rest, I.POP dst::instrs)

     | i::rest => loop(rest, i::instrs)
     {\sl ...}
\END
\end{alltt}

\section{Compiling and Running the Tool}

To compile the tool, go to the directory Tools/WhereGen, and in sml, type
\begin{alltt}
   use "build.sml"
\end{alltt}
When this completes, the image \verb|nowhere.<arch>| will be generated.

The input files to \nowhere{} should have any suffix other than \verb|.sml|.
To run the tool, type:
\begin{alltt}
    sml @SMLload=nowhere inputfile
\end{alltt}

\section{Bugs and Missing Features}
   Currently, these things are missing:
\begin{enumerate}
  \item  Array and vector patterns are currently not supported. 
        However, SML/NJ style OR-patterns are handled.
  \item Type checking of patterns is missing.
  \item The error messages may be less than clear.  Furthermore, 
        line reporting may be out-of-sync.
\end{enumerate}

\end{document}
